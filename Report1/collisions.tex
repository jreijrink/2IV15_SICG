\section{Collisions}
We implemented two different collision detections.
The first is collision between particles where particles can collide with each other.
The second is particles colliding with fixed objects, fixed object are implemented using an interface.

\subsection{Particles}
We start with iteration through all pairs of particles $(p_1, p_2)$.
For each pair the distance between them is calculated.
If the distance is less than a fixed parameter $distance$, then a collision has occurred.
Now the collision is solved:\\
\begin{eqnarray*}
    % \nonumber to remove numbering (before each equation)
     \mathit{diffpos} &=& p_1 - p_2 \\
     \mathit{normal} &=& \frac{\mathit{diffpos}}{\text{length of diffpos}} \\
     v_n &=& \mathit{normal} (\mathit{normal} \cdot v_1) \\
     v_t &=& v_1 - v_n
\end{eqnarray*}
The new velocity of $p_1$ is $v_t - (v_n * f_D)$, where $f_d$ is a damping factor.
The velocity of $p_2$ follows naturally.

\subsection{Fixed objects}
Fixed object are implemented with a \verb"Fixed Object" interface.
This interface requires a fixed object to have the following functions:
\begin{itemize}
  \item HasCollision(Particle p)\\
  \emph{ Check if a particle $p$ is colliding with the fixed object }
  \item SolveCollision(Particle p)\\
  \emph{ Fix the collision with $p$ }
\end{itemize}
Two fixed objects are implemented, a vertical and horizontal line.
If the position of a particle has changed from one side of the line to the other, a collision has occurred.
To resolve this collision the position of the particle is made equal to that of the line.
If the line is vertical, then the horizontal velocity of the particle is inverted.
We use a constant $k_r$ for the damping.
For the horizontal line the vertical velocity is inverted.

