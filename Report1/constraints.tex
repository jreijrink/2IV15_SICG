\section{Constraints}

The constraints are implemented using a generic solver.
..
\\

Each constraint is implement using the \verb"Constraint" interface.
This interface requires a constraint to have the following functions:
\begin{itemize}
  \item GetC\\
  \emph{ This function contains the constraint }
  \item GetCdot\\
  \todo { wat hier? }
  \item GetDerivative\\
  \emph{ This function contains the partial derivative of the constraint  }
  \item GetTimeDerivative\\
  \emph{ This function contains the time derivative of the constraint }
\end{itemize}

All constraints are resolved using the
To resolve all constraint the
\noindent Calculate $J$, the jacobian matrix.\\
We loop through all constraints and add their GetDerivative() values to the appropriate places in the matrix.
The resulting matrix has the structure as shown in figure \todo { verwijzing }, where $C$ is a constraint and $x$ is a particle.
\todo{ image jacobian matrix }

Calculate $W$, inverse mass \\
Calculate $\dot{J}$, jacobian time derivative matrix\\
Calculate $\dot{q}$, velocity vector\\
Calculate $Q$, constraint force vector\\
Calculate $C$, constrain vector\\
Calculate $\dot{C}$, partial derivative of constraints vector\\
$k_s$ and $k_d$ are constants used for..\\
Solve $Ax = B$ where\\
\;\;  $A = J W J^{T}$\\
\;\;  $B = (-\dot{J} \dot{q}^{T} - J W Q^{T}) - (C k_s) - (\dot{C} k_d)$\\
We added the feedback term $- (C k_s) - (\dot{C} k_d)$ to prevent drifting in the constraint force calculation.
With the calculated $x$ matrix we can calculate the constraint forces matrix using the formula $x^{T} J$.
The forces of each particles are updated using this resulting constraint forces matrix.\\

\subsection{Rod}
Used for..\\
Impl\\

\subsection{CircularWire}
Used for..\\
Impl\\

\subsection{FixedLine}
Horizontal and vertical\\
Used for..\\
Impl\\

\subsection{FixedPoint}
Used for..\\
Impl\\
