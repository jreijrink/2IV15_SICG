\section{Particles and Fluids}
We have directly ported our previous particle project in this fluid project.
The velocity field can exert forces onto the particles.
This is done by introduction two new forces, pressure and drag.

\noindent The pressure force is initialised for each particle.
At the update step the position of the particle is determined.
The density and velocity on this position are calculated using the same formula as the advection step in the fluid simulator.
The added force on the particle is $(velocity \times factor) \times density \times damp$, where $factor$ and $damp$ are constants.
The density is included so the particle will only be moved when a is fluid present.

\noindent The drag force is used to slow the particle down when there is no fluid present anymore.
At each update step the following force is added: $drag \times -velocity$, where $drag$ is a constant and $velocity$ the current velocity of the particle.

\begin{figure}[h]
    \centering
    \includegraphics[width=10cm]{img/cloth.png}
    \caption{A piece of cloth interaction with the fluid.}
    \label{fig:cloth}
\end{figure}
